\documentclass[12pt]{article}
\usepackage{graphicx,url}
\usepackage[brazil]{babel}
\usepackage[utf8]{inputenc}
\usepackage{float}
\usepackage{adjustbox}
\usepackage{amsmath}
\usepackage[margin=2.5cm,top=0.2cm]{geometry}

% Exigencias do formato:
\usepackage[T1]{fontenc}
\usepackage[utf8]{inputenc}
\usepackage{mathptmx}
\pagenumbering{gobble}% Remove page numbers (and reset to 1)

\fontsize{12}{14}
\selectfont

% Citacoes extrapolam a margem, por algum motivo.
\thispagestyle{plain}
\newcommand\spacecite{\penalty700\ \cite}

\begin{document}


\title{Explorando a Terceira Lei de Newton para \\ Simulação N-Corpos em
  uma Plataforma GPU}

\author{Vinícius Herbstrith (bolsista), Lucas Mello Schnorr (orientador) \\
Instituto de Informática -- Universidade Federal do Rio Grande do Sul}
\date{}

\maketitle

%\review{Contexto}

Simulações N-corpos são simulações de um sistema de partículas sobre a
influência das forças físicas externas e forças geradas pelas outras
partículas do mesmo sistema. Simulações deste tipo são largamente
aplicadas em astrofísica, física de plasma e dinâmica molecular.  A
simulação ocorre em intervalos de tempo onde, para cada intervalo e
para cada partícula, é calculada a força resultante que uma partícula
sofre devido a interação com as outras. Devido a necessidade de
calcular a força entre todos os pares de corpos a cada intervalo de
tempo, temos uma complexidade O($N^2$), onde N é a quantidade de
partículas presentes no sistema.
%
A simulação N-corpos pode ser otimizada fazendo-se uso da
paralelização. Como não existem dependência de dados, é possível
calcular cada corpo em paralelo, para uma complexidade de tempo O(N).
%Ainda sim, mesmo que o problema seja trivialmente paralelo, o número
%de corpos usados em simulações atinge a ordens de trilhões, fazendo
%que otimizações na paralelização sejam de grande interesse.

GPUs são uma arquitetura many-core de alto desempenho. Para se tirar
proveito deste potencial, os algoritmos geralmente precisam ser
reformulados levando em conta as peculiaridades da GPU. Uma proposta
de algoritmo para simulação N-corpos é apresentada no livro GPU Gems 3,
onde os autores extrairam um excelente desempenho da GPU.

Uma algoritmo N-corpos visando a otimização da paralelização foi
proposta por Halpem na CppCon 2014. Ele usa a terceira lei de Newton,
onde o efeito de um corpo $X$ sobre outro corpo $Y$ é idêntico ao
efeito de $Y$ sobre $X$.  Reduz-se então pela metade o número de
interações entre corpos.  O autor propõe o uso de recursão paralela
para evitar as condições de corrida do algoritmo.

O objetivo deste trabalho é a implementação deste algoritmo em uma
GPU, combinando a otimização e a capacidade do paralelismo massivo da
GPGPU. O desafio é evitar o problema de condições de corrida do
algoritmo, que não pode ser através da recursão, que apresenta um
baixo desempenho em um ambiente GPGPU. O algoritmo proposto ataca o
problema de uma forma ascendente, começando a partir do que seriam as
folhas de uma árvore recursiva. Foi necessário a adição de um esquema
mestre-trabalhador para fazer o controle das condições de corrida, tendo
em vista que a inconsistência ocorre quando dois nós de níveis
diferentes da árvore recursiva são processados em paralelo.

A análise de desempenho do algoritmo foi feita com um conjunto de
corpos gerados aleatoriamente. A máquina usada para os testes possui
processador Intel i7-4770 CPU (3.40GHz), 8GiB DIMM DDR3(1.6GHz) e uma
GPU NVIDIA GTX 760 (1152 cuda cores, 980Mhz). Como base de comparação,
usamos o algoritmo N-corpos do livro GPU gems 3 com os mesmos parâmetros
de entrada. Para diferentes tamanhos de entrada, foram feitas 15
execuções de forma se obter uma amostra significativa. Comparando os
tempos, tivemos um resultado até seis vezes mais lento com a nossa
proposta. O baixo desempenho se dá por conta do uso de mutex para
realizar o controle mestre-trabalhador, um mecanismo que acaba sendo
muito custoso no ambiente GPGPU, custo este mais alto que o
processamento reduzido dos cálculos.

Mesmo com este resultado negativo, o algoritmo é
promissor. Acreditamos que uma implementação em um ambiente em cluster
possa tirar um maior proveito das melhorias do algoritmo, sem ser
penalizado fortemente pela condição de corrida, onde podemos ter um
esquema mestre-trabalhador para um melhor controle da condição de
corrida.


\medskip
%\bibliographystyle{unsrt}%Used BibTeX style is unsrt
%\bibliography{ref}

\end{document}